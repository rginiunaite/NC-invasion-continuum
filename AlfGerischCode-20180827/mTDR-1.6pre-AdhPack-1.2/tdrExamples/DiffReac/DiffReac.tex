\documentclass{article}
\usepackage{fullpage}
\usepackage{amsmath}

\parindent0cm

\begin{document}

\begin{verbatim}
%*********************  MATLAB TDR SYSTEM  ************************************
%* File          : tdrExamples/DiffReac/DiffReac.tex
%* Date created  : 2006, January 27
%* Author(s)     : Alf Gerisch (alf.gerisch@mathematik.uni-halle.de)
%* Version       : 1.0
%* Revisions     : 
%*
%*********************  COPYRIGHT NOTICE  *************************************
%* Copyright (C) 2004-2006 Alf Gerisch
%*                         Martin-Luther-University Halle-Wittenberg
%*                         Germany
%*
%* The TDR system in Matlab has been implemented by
%*   Mathias Franz (Oct 2004 - Feb 2005)
%*   Alf Gerisch   (Oct 2004 -         )
%******************************************************************************
\end{verbatim}


This example implements the scalar linear diffusion-reaction equation
\[
\partial_t u = \nabla\cdot(D \nabla u) + g(u)
\]
with constant positive diffusion coefficient $D>0$ and reaction function
$g(u)$\,.
We always use a value $D=1$.
The reaction function can be either $g\equiv 0$, or of logistic type
$g(u)=\alpha u^\beta (1-u)$ with $\alpha,\beta>0$,
or of a generalised logistic type $g(u)=-\alpha u (1-u)(\beta -u)$.
In the generalised logistic case, $\alpha>0$ and $-1\le \beta < 1$; the case
with $0< \beta < 1$ is what the geneticists refer to as the heterozygote
inferiority case; we refer to 
[T. Kawahara and M. Tanaka. \emph{Interactions of traveling fronts: An exact
  solution of a nonlinear diffusion equation}. Physics Letters A, 97:311-314,
1983] and the references cited there for details. 

The spatial domain is either the 2D domain $(x,y)\in\Omega:=(0,1)\times(y_0,
y_0+1)$ with parameter $y_0\ge 0$ or the 3D axi-symmetric domain with cross
section $\Omega$ and the $x$-axes as axis of symmetry, that is $(x,
r(y,z))\in\Omega$. The model is used to compare these two cases.
The equation is considered for $t\ge t_0=0$.

The initial condition is always independent of $x$ and in the 2D case can be
either (\texttt{selectICandBC = 1})
\[
u(0,x,y) = (y-y_0)^2 \exp( -50(y-y_0-1)^2 )
\]
corresponding to a peak along $y=1+y_0$ and rapidly decaying towards zero for
$y\to y_0$, or  (\texttt{selectICandBC = 2})
\[
u(0,x,y) = \exp( -50(y-y_0-0.3)^2)
\]
corresponding to a peak along $y=0.3+y_0$ and rapidly decaying towards zero
for $y\to y_0$ and $y\to 1+y_0$.
In the axi-symmetric case, $y$ is simply replaced by $r(y,z)$.


In the case \texttt{selectICandBC = 1} we use constant Dirichlet boundary
conditions (equal to one) along the boundary $y=y_0+1$ in case of a 2D spatial
domain (and along the boundary $r(y,z)=y_0+1$ in case of the axi-symmetric
spatial domain). On the remaining boundary parts we use zero-flux conditions.
In the case \texttt{selectICandBC = 2} we use zeros flux condition on all the
boundary. 


The problem and its solution are independent of the $x$-coordinate. Therefor
we use in that direction a coarse grid only.


\end{document}