\documentclass{article}
\usepackage{fullpage}
\usepackage{amsmath}

\parindent0cm

\begin{document}

\begin{verbatim}
%*********************  MATLAB TDR SYSTEM  ************************************
%* File          : tdrExamples/LinearAdvection/LinearAdvection.tex
%* Date created  : 2006, January 27
%* Author(s)     : Alf Gerisch (alf.gerisch@mathematik.uni-halle.de)
%* Version       : 1.0
%* Revisions     : 
%*
%*********************  COPYRIGHT NOTICE  *************************************
%* Copyright (C) 2004-2006 Alf Gerisch
%*                         Martin-Luther-University Halle-Wittenberg
%*                         Germany
%*
%* The TDR system in Matlab has been implemented by
%*   Mathias Franz (Oct 2004 - Feb 2005)
%*   Alf Gerisch   (Oct 2004 -         )
%******************************************************************************
\end{verbatim}


This example implements the scalar linear homogeneous advection equation
\[
\partial_t u + (v_1, v_2)\cdot\nabla u = 0
\]
for $(x,y)$ in the unit square and $t>t_0=0$ with constant velocity
$(v_1, v_2)$. 
The problems solution (on the whole real plane) is given by
\[
u(t,x,y)=u_0(x-v_1t, y-v_2t)\,,
\]
where $u(0,x,y)=u_0(x,y)$ is the prescribed initial condition.

 
We consider the two cases that either $v_1=0$ or $v_2=0$ so that the
transport is along one of the coordinate directions. 
The initial condition at $t_0=0$ is
\[
u(0,x,y) = u_0(x,y):=
\begin{cases}
\max(0, 1 - 10 |0.5-x|) & \text{if } v_2 = 0\\
\max(0, 1 - 10 |0.5-y|) & \text{if } v_1 = 0\,,
\end{cases}
\]
which makes the problem essentially one-dimensional.
As boundary conditions we need to prescribe Dirichlet conditions along
the inflow boundary. We use zero Dirichlet conditions there. Along the
remaining part no boundary conditions can be prescribed. However, for
numerical purposes we simply use zero flux boundary conditions
there. This is also consistent with the above exact solution (on
unbounded domains) for sufficiently small $t\ge 0$.


The example is not yet in TDR form. To achieve such a form, we
introduce a velocity function $v(t,x,y)$ having a time independent
profile 
\[
v(t,x,y)=v_0(x,y):=x v_1 + y v_2\,.
\]
Now we can write the above problem as a TDR system with two equations
\begin{align*}
\partial_t u +\nabla\cdot(u \nabla v) &= 0\\
\partial_t v                  &= 0\,,
\end{align*}
where the initial conditions are given above. For the first equation we
use the boundary conditions as described above and for the second we
prescribe, for numerical reasons in order to evaluate derivatives
along the boundary, Dirichlet boundary conditions according to the
definition of $v$.

This TDR model is implemented. 


\end{document}