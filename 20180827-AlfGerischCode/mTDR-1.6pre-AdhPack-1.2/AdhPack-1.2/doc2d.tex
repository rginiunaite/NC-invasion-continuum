\documentclass[a4,12pt]{article}

\usepackage{fullpage}
\usepackage{amsmath}
\usepackage{amsfonts}

\newcommand{\Real}[1]{\mathbb{R}^{#1}}
\newcommand{\Toep}{\mathcal{T}}

\begin{document}


We are given a weight matrix $W$ with given structural constants $k_m, k_p,
l_m,l_p$ in the form
\[
W=
\begin{pmatrix}
k_m\backslash{}l_m&|&1&|& l_p\\\hline
1&|&&|&\\\hline
k_p&|&&|&
\end{pmatrix}
\in\Real{k_m+1+k_p, l_m+1+l_p}
\]
and a data matrix $G$
\[
G \in \Real{N_2,N_1}\,.
\]

Let 
\[
z:=k_m+1+k_p\,,\quad s:=l_m+1+l_p\,.
\]
The weight matrix $W$ is indexed either using $-k_m:1:k_p$ for the rows and
$-l_m:1:l_p$ for the columns or using $1:z$ for the rows and $1:s$ for the
columns. 
The matrix $G$ is indexed $j=1:N_2$ for the rows and $i=1:N_1$ for the columns.
We assume that the dimensions of $W$ are (much) smaller than the dimensions of
$G$.  

We can place the $(0,0)$ entry of weight matrix $W$ over an entry $(j,i)$ of $G$
and compute 
\[
a_{j,i}:=\sum_{k=-k_m}^{k_p}\sum_{l=-l_m}^{l_p}W_{k,l}G_{j+k,i+l}\,.
\]
This can be done with the given matrix $G$ if $(j,i)$ is sufficiently far away
from the border of $G$. In all other cases we must extend the data matrix $G$ to
a matrix $G_\text{ext}$ of larger dimension. 

\subsubsection*{Specific considerations for the evaluation in the
  $x_1$-direction} 

In this case we want to compute
\[
A^{(1)}=(a_{j,i})\in\Real{N_2,N_1+1}\,,\quad j=1:N_2\,, i=0:N_1\,.
\]
Therefore, the extended data matrix $G_\text{ext}^{(1)}$ must be
\[
G_\text{ext}^{(1)}=
\begin{pmatrix}
k_m\backslash{}l_m+1&|&N_1&|& l_p\\\hline
N_2&|&&|&\\\hline
k_p&|&&|&
\end{pmatrix}
\in\Real{k_m+N_2+k_p,l_m+1+N_1+l_p}\,,
\]
i.e. $G_\text{ext}^{(1)}\in\Real{N_2+z-1,N_1+s}$
The central part of $G_\text{ext}^{(1)}$ is given by the data $G$, the entries
outside that central part will be specified later.

\subsubsection*{Specific considerations for the evaluation in the
  $x_2$-direction} 

In this case we want to compute
\[
A^{(2)}=(a_{j,i})\in\Real{N_2+1,N_1}\,,\quad j=0:N_2\,, i=1:N_1\,.
\]
Therefore, the extended data matrix $G_\text{ext}^{(2)}$ must be
\[
G_\text{ext}^{(2)}=
\begin{pmatrix}
k_m+1\backslash{}l_m&|&N_1&|& l_p\\\hline
N_2&|&&|&\\\hline
k_p&|&&|&
\end{pmatrix}
\in\Real{k_m+1+N_2+k_p,l_m+N_1+l_p}\,,
\]
i.e. $G_\text{ext}^{(2)}\in\Real{N_2+z,N_1+s-1}$.
The central part of $G_\text{ext}^{(2)}$ is given by the data $G$, the entries
outside that central part will be specified later.

\subsubsection*{Considerations for the evaluation in both, the $x_1$- and the 
  $x_2$-direction} 

We are now given the weight matrix $W$, as before, as well as the extended
matrix $G_\text{ext}$ and a result matrix $A$ with dimensions
\[
G_\text{ext}\in\Real{G_z,G_s}\quad\text{and}\quad
A\in\Real{A_z,A_s}
\]
and the goal is to compute
\[
a_{j,i}=\sum_{k=-k_m}^{k_p}\sum_{l=-l_m}^{l_p}W_{k,l}G_{\text{ext},k_m+j+k,l_m+i+l}\,,\quad
j=1:A_z\,, i=1:A_s\,.
\]
All these sums are well defined since $G_\text{ext}$ has just the appropriate
size. Evaluation of these sums, however, is typically very time consuming and we
aim for a faster method of evaluation.

Since the computation of the $a_{j,i}$ is linear in the elements of
$G_\text{ext}$ we can rewrite the computation for all $(j,i)$ as a large
matrix-vector-product. To this end let $a\in\Real{A_zA_s}$ and
$g_\text{ext}\in\Real{G_zG_s}$ be the column vectors obtained by stacking up the
columns of matrices $A$ and $G_\text{ext}$, respectively. Then there exists a
matrix $T$ such that
\[
a = T g_\text{ext}\,,\quad T\in\Real{A_zA_s, G_zG_s}\,.
\]
Due to the construction of vector $a$ and $g_\text{ext}$ we consider $T$ to be a
block matrix with blocks $T_{J,I}$,
\[
T= (T_{J,I})\,,\quad T_{J,I}\in\Real{A_z,G_z}\,,\quad J=1:A_s\,, I=1:G_s\,.
\]
It will turn out, that each block $T_{J,I}$ is a Toeplitz matrix and that $T$ at
the block level has Toeplitz structure, too, i.e. $T$ is a block-Toeplitz matrix
with Toeplitz blocks. We specify the blocks in the following but first introduce
a notation.

A Toeplitz matrix is characterised by its first column and first row
entries. The Toeplitz matrices which we will encounter are banded with a lower
band width $l$ and an upper bandwidth $u$ and so only $l$ elements of the first
column and $u$ elements of the first row must be specified. This is done by the
following notation (which gives the main diagonal entry twice):
\[
\Toep(t_{-l},t_{-l+1},\dots,t_{-1},t_0 \mathbf{;} t_0, t_1, \dots, t_u)\,.
\]
Note the semicolon.
Now it is easily verified that the $(1,1)$ block of $T$ is
\[ 
T_{1,1}=\Toep(w_{1,1}; w_{1,1},w_{2,1}, \dots, w_{z,1})=:T_1\,.
\]
Similarly follows for the next blocks in the first row
\[ 
T_{1,I}=\Toep(w_{1,I}; w_{1,I}, w_{2,I}, \dots, w_{z,I})=:T_I\in\Real{A_z,G_z}\,,
\quad I=1:s
\]
and then
\[
T_{1,I}=0\in\Real{A_z,G_z}\,,\quad I=s+1:G_s\,.
\]
Summarising, we obtain the following $G_s$ blocks in the first block row
\[
T_1, T_2,\dots, T_s,0,\dots,0\,.
\]
In the second block row we have the following $G_s$ blocks
\[
0, T_1, T_2,\dots, T_s, 0,\dots, 0\,.
\]
This continues until we obtain in the last block row the $G_s$ blocks
\[
0,\dots,0,T_1, T_2,\dots, T_s\,.
\]
This demonstrates the block-Toeplitz structure of matrix $T$.

We now embed matrix $T$ in a suitable block-circulant matrix with circulant
blocks. We start at the level of the individual blocks. Each Toeplitz block
$T_{J,I}\in\Real{A_z,G_z}$ has lower bandwidth $l=0$ and upper bandwidth
$u=z-1$. Hence it can be embedded (as upper left block) in a circulant matrix of
dimension
\[
\ell=\max\{A_z+u,G_z+l\}= \max\{A_z+z-1,G_z\}\,.
\]
The circulant matrix $\hat{T}_I\in\Real{\ell,\ell}$, $I=1:s$, corresponding to
the Toeplitz matrix $T_I$, is now defined by
its first column, which is given by
\[
\hat{t}_I=(w_{1,I},0,\dots,0,w_{z,I}, w_{z-1,I},\dots,w_{2,I})^T
\in\Real{\ell}\,, \quad I=1:s\,.
\]
The zero blocks in $T$ are increased to size $\ell\times\ell$, too, such that we
obtain a matrix $\hat{T}\in\Real{\ell A_s,\ell G_s}$.
Matrix $G_\text{ext}$ must be extended to $\hat{G}_\text{ext}\in\Real{\ell,G_s}$
by adding the required number of zero rows at the bottom end. 
The result matrix $A$ becomes also
larger and we have $\hat{A}\in\Real{\ell,A_s}$ and the first $A_z$ rows of
$\hat{A}$ equal the matrix $A$. Again, there is a correspondence between the
matrices $\hat{G}_\text{ext}$ and $\hat{A}$ and corresponding vectors
$\hat{g}_\text{ext}$ and $\hat{a}$, respectively. 

What we have done at the level of each block of $T$, can now be performed at
the block-level of matrix $\hat{T}$. The matrix $\hat{T}$ is a
block-Toeplitz matrix with $A_s\times G_s$ blocks and has lower (block)
bandwidth $l=0$ and upper (block) bandwidth $u=s-1$.
Hence it can be embedded (as the upper left part) in a block-circulant matrix 
$\bar{T}$ with $L\times L$ blocks, where
\[
L=\max\{A_s+u,G_s+l\}= \max\{A_s+s-1,G_s\}\,.
\]
As a block-circulant matrix, the matrix $\bar{T}$ is described by its $L$ blocks in
the first column given by
\[
\hat{T}_1, 0, \dots, 0, \hat{T}_s, \hat{T}_{s-1}, \dots, \hat{T}_2\,.
\]
Since each of these blocks itself is circulant, the matrix $\bar{T}$ can be
described by the first columns of these blocks, which are conveniently arranged
in the following matrix
\[
V:=
[\hat{t}_1, 0, \dots, 0, \hat{t}_s, \hat{t}_{s-1}, \dots,
\hat{t}_2]\in\Real{\ell,L}\,. 
\] 
Extended matrices $\bar{A}, \bar{G}_\text{ext}\in\Real{\ell,L}$ are obtained from
$\hat{A}$ and $\hat{G}_\text{ext}$, respectively, by adding zero columns at the
right end. The vectors $\bar{a}$ and $\bar{g}_\text{ext}$ are defined as before
and it then holds
\[
\bar{a} = \bar{T}\bar{g}_\text{ext}\,,
\]
where $\bar{T}$ is block-circulant with circulant blocks. Hence this product can
be conveniently and efficiently computed via 2D-FFT as
\[
\bar{A} = \text{iFFT2}(\text{FFT2}(V) \otimes \text{FFT2}(\bar{G}_\text{ext}))\,.
\]
The object of interest, matrix $A$ can be recovered as the upper left block of
matrix $\bar{A}$.
\end{document}

